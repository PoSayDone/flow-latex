\subsection{Клиентская часть}

Реализация клиентской части началась с создания всех основных элементов интерфейса: кнопок, элементов поиска и прочего.

Далее было решено перейти к созданию страниц и логики форм.

\subsubsection{Главная страница}
Первой стала главная, ее итоговый вид представлен на рисунке \ref{flow-main}.

\image{flow-main}{Главная страница}{0.3}

На главной странице имеются карточки пользователей и кнопки для действий. При загрузке главной страницы, мы загружаем данные пользователя, список всех возможных интересов, целей поездки, отправлений, прибытий. Для загрузки данных, мы используем функцию load из SvelteKIT. Данная функция выполняет указанный нами код при загрузке страницы. В нашем случае это выполнение запросов. Код функции представлен ниже.

\codefromfile{main-load.ts}{JavaScript}

При нажатии кнопки "нравится" или "не нравится", а так же при пролистывании карточки влево или вправо на сервер отправляется запрос на добавление пользователя с карточки в таблицу "matches".

Так же с главной страницы мы можем получить доступ к модальному окну изменения статуса. Модальное окно представлено на рисунке \ref{flow-status-modal}.

\image{flow-status-modal}{Модальное окно изменения статуса}{0.3}

Данное окно является формой, в которой пользователь выбирает текущий статус (активный, неактивный), если пользователь активен, то он может изменить свои цели поездки, выбрать места из которых он готов отправиться и соответственно куда готов отправиться \cite{svelte-superforms}.

Всего пользователь может выбрать не более 3 локаций из откуда и 5 локаций куда, и там, и там, пользователь может выбрать вариант все, тогда будет считаться, что пользователь выбрал все возможные варианты. Целей поездки можно выбрать не более 3.

Для реализации логики форм, используется пакет <<SuperForms>>, он позволяет легко реализовать валидацию данных, вывод ошибок при неправильном заполнении форм и так же упрощает отправку данных из формы на сервер.

При закрытии формы нажатием на кнопку готово или при нажатии за пределами модального окна, данные из формы отправляются на сервер, при помощи инстурмента form actions из SvelteKIT \cite{svelte-form-actions}. Плюс использования actions, в том, что даже если JavaScript у пользователя будет выключен, они будут продолжать работать. Код запроса представлен ниже.

\codefromfile{status-action.ts}{JavaScript}

\subsubsection{Авторизация}

Страница авторизации представлена на рисунке \ref{flow-auth}.

\image{flow-auth}{Страница авторизации}{0.5}

Мы имеем простую форму для авторзации, всего 2 поля, при подтверждении мы вызываем action входа. Код представлен ниже.

\codefromfile{auth-action.ts}{JavaScript}

\subsubsection{Чаты}

Итоговый вид страницы чатов представлен на рисунке \ref{flow-chats}.

\image{flow-chats}{Страница чатов}{0.5}

На данной странице отображаются все чаты пользователя, мы получаем их из load функции. На данной странице мы подключаемся к <<Pusher>> и ожидаем сигнала на появление нового чата или обновления старого. Если такое происходит, мы вызываем функцию invalidateAll, которая перезагружает данные из load функции и мы получаем обновленный данные о чатах \cite{svelte-invalidateall}.

\subsubsection{Чат с пользователем}

Итоговый вид страницы представлен на рисунке \ref{flow-chat}.

\image{flow-chat}{Диалог пользователей}{0.5}

При загрузке данной страницы, мы берем используем функцию load и делаем запрос на сервер, чтобы получить данные о данном диалоге. После того как мы получаем данные, создаем элементы сообщений.

Так же мы подключаемся к <<Pusher>> и ожидаем новых сообщений другого пользователя. Если приходит новое сообщение, мы добавляем его в массив сообщений из которого формируются HTML элементы сообщений. Код подключения к <<Pusher>> представлен ниже.

\codefromfile{chat-pusher.ts}{JavaScript}

Поле ввода является формой, при ее подтверждении мы вызываем form action, для отправки запроса на добавление сообщения на сервер. Код представлен ниже.

\codefromfile{chat-server.ts}{JavaScript}

\subsubsection{Профиль}

Итоговый вид страницы профиля представлен на рисунке \ref{flow-profile}.

\image{flow-profile}{Страница профиля}{0.5}

Данная страница является самой нагруженной по формам. В сумме здесь 4 формы:
\begin{itemize}
    \item Форма статуса из шапки;
    \item форма с данными пользователя;
    \item форма с интересами пользователя;
    \item форма с изображением профиля.
\end{itemize}

Форма с данными пользователя самая простая, она включает в себя 4 поля ввода и при потере фокуса на любом из полей или при переходе на другую страницу, отправляет запрос на обновление данных пользователя.

Форма с интересами представляет собой список всех возможных интересов, пользователь может выбрать до 5 интересов. Форма представлена на рисунке \ref{flow-interests-modal}.

\image{flow-interests-modal}{Форма изменения интересов}{0.5}

Форма с изображением профиля имеет 2 кнопки, для редактирования и удаления фотографии профиля. Она представлена на рисунке \ref{flow-avatar-modal}.

\image{flow-avatar-modal}{Форма изменения изображения профиля}{0.5}

При загрузке фото, появляются 2 другие кнопки для подтверждения или отмены загрузки фото. Вид данного окна представлен на рисунке \ref{flow-avatar-modal-picked}.

\image{flow-avatar-modal-picked}{Форма изменения изображения профиля}{0.5}

Все эти формы так же используют <<SuperForms>> и form actions. Код представлен ниже.

\codefromfile{profile-actions.ts}{JavaScript}

\subsubsection{Профиль другого пользователя}

Итоговый вид страницы профиля другого пользователя представлен на рисунке \ref{flow-profile-other}.

\image{flow-profile-other}{Страница пользователя}{0.45}

При загрузке страницы мы запрашиваем данные с сервера при помощи функции load и id пользователя. Код представлен ниже.

\codefromfile{profile-other-load.ts}{JavaScript}
