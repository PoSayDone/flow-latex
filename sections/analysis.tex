\section{Анализ предметной области для формирования требований}

Анализ решено было решено разбить на 3 части:
\begin{enumerate}
    \item Подтверждение практической значимости и надобности в продукте.
    \item Интервью с предполагаемыми пользователями и формирование сценариев взаимодействия.
    \item Анализ конкурентов.
    \item Макетирование интерфейса.
\end{enumerate}

Все этапы анализа должны проходить поочередно, потому что каждый последующий этап зависит от результата предыдущего.

\subsection{Обоснование актуальности и практической значимости}
Анализ было решено начать с подтверждения надобности пользователей в реализации продукта. Для этого воспользовался сервисами Google Trends и Яндекс ВордСтат, которые предлагают получить статистику поисковых запросов по ключевым словам.

Результаты запроса <<Поиск попутчиков на отдых>> представлены на рисунке 1.1.

\image{wordstat}{Результаты Яндекс ВордСтат}{0.8}

Google Trends не дает доступа к информации о количестве запросов, поэтому его данные не были учтены в анализе, так как не представляется возможным как-либо адекватно интерпретировать данные полученные от этого сервиса. Яндекс ВордСтат в свою очередь предоставляет полные метрики по запросам.

В среднем количество запросов в месяц составляет около 6 тысяч, самые пиковые значения приходятся на июнь, июль и август, периоды, когда люди больше отправляются в путешествия.

Из полученных данных мы понимаем, что на рынке существует потребность в подобном продукте и люди регулярно ищут подобные сервисы.

Еще одним подтверждением актуальности и практической значимости  будущего продукта, являются отзывы пользователей похожих продуктов на площадках Google Play и App Store. Отзывы пользователей не только демонстрируют потребность в подобном продукте, но также показывают, что существующие продукты не закрывают потребности пользователей в связи со своей недоработанностью, малой функциональностью и плохой работоспособностью. Комментарии пользователей приложений конкурентов представлены на рисунках с \ref{comment_1} по \ref{comment_6}.

\image{comment_1}{Комментарий пользователя}{0.8}
\image{comment_2}{Комментарий пользователя}{0.8}
\image{comment_3}{Комментарий пользователя}{0.8}
\image{comment_4}{Комментарий пользователя}{0.5}
\image{comment_5}{Комментарий пользователя}{0.5}
\image{comment_6}{Комментарий пользователя}{0.5}



\subsection{Интервью с пользователями}

Далее были проведены интервью с потенциальными пользователями будущего приложения. На основе проведенных интервью были сформированы сценарии взаимодействия с приложением на основе фреймворка Jobs To Be Done. Cценарии взаимодействия представлены в таблице \ref{jtbd}.

\begin{table}[H]
    \raggedright
    \caption{Jobs To Be Done}\label{jtbd}
    \begin{tabularx}{\textwidth}{|X|X|X|}
        \hline
        \multicolumn{1}{|c|}{Ситуация} & \multicolumn{1}{c|}{Мотивация} & \multicolumn{1}{c|}{Желаемый результат} \\ \hline
        Когда я хочу отправиться в путешествие в компании, но не имею человека, который сможет поехать со мной &
        Я хочу найти человека или компанию, которые бы разделяли мои интересы и ценности &
        Чтобы поездка была наполнена положительными эмоциями, и я нашел новые знакомства \\ \hline
        Когда я хочу отправиться в путешествие в компании, но не имею человека, который сможет поехать со мной &
        Я хочу найти человека или компанию, которые бы разделяли мои интересы и ценности в современном формате и не просматривать кучу странных объявлений &
        Чтобы я быстро и весело нашел себе попутчика \\ \hline
        Когда я хочу поехать в путешествие, но не имею большого бюджета для него &
        Я хочу найти человека или компанию, с которыми можно разделить часть расходов &
        Чтобы путешествие получилось бюджетным, но еще более веселым и запоминающимся \\ \hline
    \end{tabularx}
\end{table}

По итогам проведенных интервью были выделены 3 задачи, которые пользователи хотят выполнить при помощи приложения по поиску попутчиков в путешествия. Эти задачи помогут смоделировать интерфейс приложения максимально лаконичным и удобным именно под нужды пользователей.

\subsection{Анализ конкурентов}

Следующим этапом стал анализ конкурентов. Для этого было выбрано 6 приложений прямых и косвенных конкурентов, были выделены все экраны для того, чтобы проанализировать функционал имеющийся в этих приложениях.

\image{screens}{Экраны прямых и косвенных конкурентов}{0.8}

Приложения <<Погнали>>, <<Team2Travel>> и <<Lovoyage>> имеют схожие экраны  главной страницы, она выглядит как доска объявлений со списком всех доступных пользователей. В остальном экраны приложений схожи и обладают похожим функционалом.

Решение с доской объявлений не самое подходящее для такого приложения, так как мы живем в <<эпоху рекомендательных алгоритмов>>, где буквально в каждом приложении или на каждом сайте есть система рекомендаций на основе каких-то данных о пользователе, в пример можно привести TikTok, самую популярную социальную сеть на данный момент, которая поставила рекомендательные алгоритмы во главу приложения и стало одной из ключевых причин его успеха.

Также в пример можно привести приложение для знакомств <<Tinder>>, которое тоже обладает большой популярностью. Там рекомендательные алгоритмы <<решают>> какой пользователь попадется вам в карточку на основе статистики.

В нашем же случае мы возьмем модель приложения <<Tinder>>, так как оно доказало успешность такого решения своей популярностью. Пользователь будет указывать свои интересы и цели на поездку, а уже на основе этих данных пользователю будут предоставляться рекомендуемые кандидаты.

\subsection{Макетирование интерфейса}

Далее была сформирована схема user flow, см. ПРИЛОЖЕНИЕ \ref{pril:user-flow}, ней изображен путь пользователя по приложению. Данная схема позволит спроектировать дизайн приложения, не забыв учесть какие-либо функции, а также позволит наглядно изобразить архитектуру будущего приложения.

Итогом проведенного анализа, стал макет приложения реализованный при помощи инструмента Figma. Общий вид спроектированных страниц представлен на рисунке \ref{figma}.

\image{figma}{Готовые страницы в Figma}{0.8}

Были смоделированы страницы входа и регистрации, главная, чат, профиль пользователя, и модальные окна изменения интересов и целей поездки. Примеры страниц представлены на рисунке \ref{screens-flow}.

\image{screens-flow}{Экраны приложения}{0.8}
