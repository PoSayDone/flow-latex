\section*{Введение}
\addcontentsline{toc}{section}{Введение}

В повседневной жизни люди всё больше и больше начали полагаться на общение и знакомства через социальные сети, мессенджеры и другие интернет ресурсы \cite{social-media-users}. Поэтому и поиск новых знакомств также перетекает в интернет.

Так-же в России ежегодно растет доля внутреннего туризма \cite{vnut-turism}.

Именно эти тренды были основополагающими при выборе темы моего прикладного проекта. Мной было решено совместить эти два тренда в одну идею. Этой идеей стало веб-приложение для поиска попутчиков для путешествий.

Такое приложение подстегнет людей, у которых нет знакомых, с которыми можно отправиться в путешествие, к путешествиям, а также позволит найти новые знакомства.

Так как пользователи мобильных устройств составляют 80\% времени проведенного в социальных сетях \cite{mobile-vs-desktop-usage}, то было принято решение ориентироваться на мобильные устройства.

Для упрощения разработки под все типы мобильных устройств, решено использовать технологию Progressive Web App (далее PWA). Технология была представлена Google для упрощения создания мультиплатформенных приложений. Технология предоставляет набор инструментов, который позволяет превратить сайт, в нативное приложение \cite{what-are-pwas}.

Объектом автоматизации являются человеческие отношения.

Цель работы – создание информационной системы обеспечивающей пользователям удобный инструмент для поиска попутчиков, с целью совместных путешествий.
