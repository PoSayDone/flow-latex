\defaultfontfeatures{Ligatures=TeX}
\setdefaultlanguage[spelling=modern]{russian}
\setotherlanguage{english}
\setmainfont{Times New Roman}
\setmonofont[Scale=MatchLowercase]{DejaVu Sans Mono}
\setmonofont[Scale=0.9]{DejaVu Sans Mono}
\newfontfamily\cyrillicfont{Times New Roman}

\eqexplSetDelim{\textendash~}
\renewcommand\labelitemi{\textendash~}
\usepackage[left=3cm,right=1.5cm,top=2cm,bottom=2cm]{geometry}
\usepackage[fontsize=13pt]{scrextend}

%Пояснения к формулам
\eqexplSetIntro{где}

\setlist[enumerate]{label*=\arabic*.}

% уменьшаем отступы у формул
\expandafter\def\expandafter\normalsize\expandafter{%
    \normalsize%
    \setlength\abovedisplayskip{0pt}%
    \setlength\belowdisplayskip{0pt}%
    \setlength\abovedisplayshortskip{0pt}%
    \setlength\belowdisplayshortskip{0pt}%
}

\setlength{\parindent}{1.25cm} % отступ для абзаца
\onehalfspacing

\AddToHook{cmd/section/before}{\clearpage} % Каждый section будет начинаться с новой страницы

\titleformat*{\section}{\fontsize{16}{16}\bfseries\centering}
\titleformat*{\subsection}{\fontsize{14}{14}\bfseries\centering}
\titleformat*{\subsubsection}{\fontsize{13}{13}\bfseries\centering}
\titleformat{\chapter}[display]{%
    \normalfont\huge\bfseries%
}{%
    \chaptertitlename\ \thechapter%
}{%
    0pt% Space between "Chapter 1" and "Introduction"
}{%
    \Huge%
}
\titlespacing*{\section}
{0pt}{0pt}{12pt}
\titlespacing*{\subsection}
{0pt}{12pt}{6pt}
\titlespacing*{\subsubsection}
{0pt}{8pt}{4pt}

\graphicspath{ {./images/} }
\counterwithin{figure}{section}
\DeclareCaptionFormat{figure-format}{\fontsize{12}{14}\bfseries\itshape\selectfont#1#2#3}
\captionsetup{skip=0pt}
\captionsetup[figure]{format=figure-format, name=Рисунок, labelsep=endash, aboveskip=2pt}

% Настройки для содержания
\renewcommand{\cftsecleader}{\cftdotfill{\cftsubsecdotsep}}
\renewcommand{\cfttoctitlefont}{\fontsize{16}{0}\bfseries\centering\hfil}% Remove \bfseries from ToC title
\renewcommand{\cftsecfont}{}% Remove \bfseries from section titles in ToC
\renewcommand{\cftsecpagefont}{}% Remove \bfseries from section titles' page in ToC
\setlength{\cftbeforesecskip}{0pt}
\emergencystretch=25pt

\setlist{noitemsep, topsep=0pt, leftmargin=2cm}
\setlist[2]{noitemsep, topsep=0pt, leftmargin=*}

\sloppy				% Избавляемся от переполнений
\clubpenalty=10000		% Запрещаем разрыв страницы после первой строки абзаца
\widowpenalty=10000		% Запрещаем разрыв страницы после последней строки абзаца

\captionsetup[table]{labelsep=endash, justification=raggedright, singlelinecheck=off} %установки для заголовков таблиц

% Номера страничек
\urlstyle{same}
\pagestyle{fancy}
\renewcommand{\headrulewidth}{0pt}
\fancyhead{}
\fancypagestyle{plain}{
    \fancyhead{}
    \fancyfoot[C]{\normalfont\thepage}
}

\renewcommand\tabularxcolumn[1]{m{#1}}% for vertical centering text in X column

\renewcommand{\tabcolsep}{6pt}
\renewcommand{\arraystretch}{1.4}


% Приложения
\makeatletter
\renewcommand{\appendixname}{Приложение}
\newcommand\appendix@section[1]{%
  \renewcommand{\thesection}{\Asbuk{section}}%
  \refstepcounter{section}%
  \orig@section*{\centering\MakeUppercase{\appendixname}~\thesection\\ #1}%
  \addcontentsline{toc}{section}{\appendixname~\thesection.~~#1}%
}
\let\orig@section\section
\g@addto@macro\appendix{\let\section\appendix@section}
\makeatother

\let\oldtabularx\tabularx
\renewcommand{\tabularx}{\small\oldtabularx}

\setlength{\intextsep}{12pt}

%Листинги
\definecolor{codegreen}{rgb}{0,0.6,0}
\definecolor{codegray}{rgb}{0.5,0.5,0.5}
\definecolor{codepurple}{rgb}{0.58,0,0.82}

\lstdefinestyle{mystyle}{
    commentstyle=\color{codegreen},
    keywordstyle=\color{magenta},
    numberstyle=\tiny\color{codegray},
    stringstyle=\color{codepurple},
    basicstyle=\ttfamily\footnotesize,
    breakatwhitespace=false,
    breaklines=true,
    captionpos=b,
    keepspaces=true,
    numbers=left,
    numbersep=5pt,
    showspaces=false,
    extendedchars=\true,
    showstringspaces=false,
    showtabs=false,
    tabsize=2,
    float=H,
}

\lstset{style=mystyle}
