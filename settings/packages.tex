\usepackage{amsmath}
\usepackage{setspace}
\usepackage{polyglossia} % языковой пакет
\usepackage{fontspec}
\usepackage{hyperref}
\usepackage[compact]{titlesec}
\usepackage{graphicx} % пакет для использования графики (чтобы вставлять рисунки, фотографии и пр.)
\usepackage{float}
\usepackage{caption}
\usepackage{tocloft}
\usepackage{titletoc}
\usepackage{enumitem}
\usepackage{eqexpl}
\usepackage{color}
\usepackage{tabularray}
\usepackage{tabularx}
\usepackage{listings}
\usepackage{url}
\usepackage{fancyhdr}
\usepackage[%
backend=biber,          % Движок
bibencoding=utf8,       % Кодировка bib-файла
sorting=none,           % Настройка сортировки списка литературы
style=gost-numeric,     % Стиль цитирования и библиографии по ГОСТ
language=auto,          % Язык для каждой библиографической записи задается отдельно
autolang=other,         % Поддержка многоязычной библиографии
sortcites=true,         % Если в квадратных скобках несколько ссылок, то отображаться будут отсортированно
movenames=false,        % Не перемещать имена, они всегда в начале библиографической записи
maxnames=5,             % Максимальное отображаемое число авторов
minnames=3,             % До скольки сокращать число авторов, если их больше максимума
doi=false,              % Не отображать ссылки на DOI
isbn=false,             % Не показывать ISBN, ISSN, ISRN
]{biblatex}
\DeclareDelimFormat{bibinitdelim}{}

